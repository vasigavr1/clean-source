\section{Discussion} \label{sec:disc}
In this work, we have used the \rts\ to model the protocol, allowing us to create a mapping from MCMs to protocols. In this section, we discuss other use-cases of the \rts. We start by describing the relation of \srts\ to linearizability and then we discuss  how they can be used to describe the real-time guarantees of existing protocols and to decide the consistency guarantees provided when protocols are composed.

\subsection{Relation of \srts\ to linearizability}\label{disc:lin}
Formally, the lin criterion~\cite{Herlihy:1990} can be written in our notation as follows.
An execution \Exec\ is linearizable if there exists $rl \in RL$ such that $hb \subseteq rl$. Furthermore, because lin is a \emph{local} property, it suffices that only operations to the same object abide by the rule. Plainly, for two operations $a,b$ to the same object $x$, if $(a,b) \in hb$ then it must be that $(a,b) \in rl$. This is equivalent to: $acyclic(hb, syn)$.
This, in turn, is equivalent to enforcing all four \srts.
Therefore, the lin property is the property of enforcing all four \srts.


\beginbsec{Intuition}
Informally, the lin property can be defined by writing that \emph{each memory operation appears to take effect at some point between its invocation and completion}~\cite{Herlihy:2008}.
Notably, each of the \srts\ is a specialization of this definition.
For instance, $\rtwr$ mandates that each write appears to take effect at some point between its invocation and completion w.r.t. every read.
Combining all four \srts\ mandates that each write or read appears to take effect at some point between its invocation and completion w.r.t. every read or write. This is equivalent to the lin definition.

\subsection{\Srts\ for compositionality}\label{disc:cons}

So far we have viewed the \srts\ solely as a mathematical model of the protocol.
However, in distributed systems, it is often necessary to describe the real-time guarantees offered by a system, in order to compose different systems.
Consider the example of Zookeeper. Zookeeper does not offer linearizability, and thus multiple Zookeeper instances cannot be composed. However, researchers have found that Zookeeper does offer some real-time guarantees that can be leveraged to achieve compositionality~\cite{LevAri:2016}.

\Srts\ can be used to capture these real-time guarantees. For example, the precise real-time guarantees of Zookeeper are the $\rtww$ and the $\rtrw$ \srts and all four \prts.
More importantly, given this knowledge we can specify the MCM provided by composing different Zookeeper instances: the composed system will enforce the same \rts\ as a single Zookeeper instance and thus we can use the reverse mapping from \rts\ to \synpats\ to specify the resulting MCM. In fact, we can use \srts\ in the same spirit, to specify the MCM of any combination of composed systems, by asserting that the composed system will enforce any \rt\ that is enforced by all of its subsystems.






