\section{From \rts\ to Protocols} \label{sec:prot}

In previous sections we established the mappings between the \synpats\ and the \rts. 
In this section, we relate \rts\ to some well-known protocol design techniques. %
We start with a brief discussion on the \prts\ and then we focus on the \srts.






\subsection{Enforcing \rts}
\Prts\ model the operation of the \rob\ specifying  when the memory system can begin executing an operation.
Upholding the \prts\ is as simple as inspecting the state of the \rob.
For instance, enforcing $\prtwr$ implies that the memory system cannot begin executing a read $r$ from process $p$, until every preceding write in the \rob\ is completed.



\custvspace
\Srts\ models how the memory system executes reads and writes.
Below we discuss two common techniques that can be used to enforce \srts\ 1) \emph{overlap} and 2) \emph{lockstep}.

\beginbsec{Overlap}
The \srt\ $srt_{mn}$ can be enforced simply by ensuring that operations of type $m$ must overlap with operations of type $n$ in a physical location. For instance, we can enforce $\rtwr$, by ensuring that a write is propagated to $x$ nodes and a read queries $y$ nodes, where $x+y > N$ and $N$ is the number of nodes. 
Alternatively, both types of operations can \qt{meet} in some centralized physical location (\eg the directory for multiprocessors). 
To ensure all four \srts\ and thus linearizability, both reads and writes must query $y$ nodes to learn about completed operations and must broadcast their results to $x$ nodes. This is exactly how the multi-writer variant of ABD~\cite{Lynch:1997} operates.

\beginbsec{Lockstep}
Lockstep is %
a technique, %
where a memory system node first \qt{grabs a lock} on the object before beginning an operation and releases it when the operation completes. 
Upon grabbing the lock, the node learns about the operation executed by the previous lock holder.
The act of \qt{grabbing the lock} is similar to getting a cache-line in $M$ or $S$ state in a coherence protocol~\cite{Vijay:2020}, or becoming the leader of the next log entry in a state machine replication protocol, such as Paxos~\cite{Lamport:1998}. Notably, lockstep entails overlap as operations must meet in a physical location to exchange the lock, but it also precludes the operations from executing concurrently.

There are two aspects of lockstep that can enforce \srts.
First, the \srt\ between two operations is enforced if a lock must be passed from one to the other. 
For example we can enforce $\rtww$ by mandating that a lock must be grabbed to perform a write.
Second, locking also ensures that certain operations cannot overlap in real-time.
When a write cannot overlap with a write and $\rtww$ is enforced then $\rtrw$ is also enforced. This is because if a read $r$ returns the value of write $w$, then a write $k$ that begins after $r$ has completed must also begin after $w$ has completed.
Similarly, when a write cannot overlap with a read and $\rtwr$ is enforced, then $\rtrr$ is enforced. This is because if a read $r$ returns the value of write $w$, then it must be that $w$ completes before $r$ begins. Therefore, a read $m$ that begins after $r$ has completed, must also begin after $w$ has completed and thus will observe $w$.
Protocols often combine the two aspects of lock-step to enforce the single-writer multiple-reader invariant (\SWMR)~\cite{Vijay:2020}, where for any given object at any given time there is either a single write in progress or multiple reads.
This ensures that writes must grab a lock from each other ($\rtww$), reads must grab a lock from writes ($\rtrw$), writes cannot overlap in time ($\rtrw$) and writes do not overlap in time with reads ($\rtrr$).




