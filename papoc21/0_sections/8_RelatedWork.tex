\section{Related Work} \label{sec:related}



This work is the first to provide a mapping from \mcms\ to the protocols that can enforce them. To present this mapping we have used an abstract system model and the formalism presented by Alglave \etal~\cite{Alglave:2014} in order to describe \mcms, executions and real-time guarantees. Several works~\cite{Szekeres:2018,Crooks:2017, Burckhardt:2014, LevAri:2017, Gotsman:2017} have also described similar system models and formalism, but differ from our work in that they do not provide a mapping from \mcms\ to the protocols.
Specifically, Szekeres and Zhang~\cite{Szekeres:2018} provide a system model and a formalism called \emph{result visibility} to describe consistency guarantees, including real-time guarantees.
Crooks \etal\cite{Crooks:2017}, focusing on databases, provide a state-based formalization of isolation guarantees.
Burckhardt, in his book on Eventual Consistency~\cite{Burckhardt:2014}, provides a formalism to describe consistency models and protocols, with a focus on weaker guarantees.
Lev-ari \etal~\cite{LevAri:2017} define Ordered Sequential Consistency, OSC(A), in order to specify the real-time guarantees of a protocol (with a focus on Zoo\-kee\-per~\cite{Hunt:2010}). 
Similarly, Gotsman  and  Burckhardt propose GSC~\cite{Gotsman:2017}, a generic operational model for systems that totally order all writes, which can capture all of the \srts\ for such systems.


\custvspace
In the cache coherence literature, the four \emph{program-orderings} are used to describe consistency guarantees~\cite{Vijay:2020}. In fact, 
researchers have shown that when the coherence protocol enforces the \emph{single-writer multiple-reader} (\SWMR) invariant, the \mcm\ depends solely on the enforced program orderings~\cite{Arvind:2006, Meixner:2005}.
Program orderings are very similar to \prts\ with the subtle difference that program orderings carry the implication that the memory system enforces \SWMR. In contrast, \prts\ make no such assumption, allowing us to explore all possible behaviours of the memory system.

Finally, CCICheck~\cite{Manerkar:2015} provides a way to verify an existing coherence protocol against its target \mcm\ using the notion of μhb orderings, which are related to real-time.
Our work provides a mapping from \mcms\ to protocols, enabling the design of minimal protocols for any \mcm.























