\section{Conclusion and Lessons Learned}
\label{sec:conclusion}
% \FloatBarrier{}

The goal of the paper is to uncover the impact of modern hardware on the performance of strongly-consistent replication protocols.
To this end, we presented \odlib, a framework that enables the fast development and deployment of replication protocols over modern hardware. 
Over \odlib, we built and evaluated \pnum\ protocols. 
Extrapolating their results to the entire design space through an informal taxonomy, we provided a characterization of strongly-consistent replication protocols.

%\beginbsec{Lessons learned}
% Let us summarize the lessons learned. %distill
On the system side, we experienced first-hand the necessity for a reliable, high-performance framework to design, build and deploy replication protocols. Without it, system-level bugs (networking, KVS etc.) become a black hole for developer time.
In hindsight, this is no surprise: clean interfaces that abstract orthogonal components have been the cornerstone of computer science. %, and enables one to focus on the important stuff. %expertise,
%, is well established.
Nevertheless, 
% \antonis{May replace "Nevertheless" w/ "With that in place" or smthing better} 
we were pleasantly surprised to see that we can build and deploy a new protocol in two days (\S\ref{sec:why}). 
%Again, in hindsight, there is no reason why a simple protocol with a clean specification should take more time. Yet, we were pleasantly surprised.

When it comes to protocol design, the overarching lesson is that the true limits of a protocol will be uncovered only when all artificially imposed bottlenecks have been removed. 
Plainly, this calls for highly-optimized, multi-threaded and \RDMA-enabled implementations. 
It is very telling that ZAB outperforms Classic Paxos (CP) by more than 2x when both are single-threaded, but the result is inverted when they are multi-threaded. 
The pseudo bottleneck of single-thread implementations conceal ZAB's inefficiencies while holding back CP's capabilities. Multi-threading removes the bottleneck, laying bare the true nature of the protocols.
%bare, once the bottleneck is removed.
% Multi-threading the protocols lays their true nature


% This means that we need to invest effort on stressing he hardware multi-threading and 

% Without the framework, the developer will most likely spent the overwhelming majority of her time dealing with bugs in the RDMA networking layer. 
% dealing with bizarre design choices and delphian user manuals of 
% Firstly, our characterization demonstrated the importance of investing on multi-threading in the modern era.
% and it showcased that multi-threading changes the relative

% thread-scalability in the modern era multi-threading the protocol implementations
% demonstrated the relative impact of thread-scalability, load balance and the work-per-request ratio and shed light on how such properties are affected by operational patterns such as totally ordering all writes or having a leader. 