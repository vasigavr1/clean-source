\section{Related Work}
\label{sec:related}

\beginbsec{Related Frameworks}
% There are two alternatives to the functionality offered by \odlib.
Similarly to \odlib, 
Paxi~\cite{Ailijiang:2019} offers a rich interface that enables the fast development of replication protocols. However, Paxi is neither multi-threaded nor \RDMA-enabled.
eRPC~\cite{Kalia:2019} is a general-purpose networking framework offering \RDMA-based RPCs, similarly to \odlib.
However, \odlib\ also provides functionality tailored for replication protocols, such as the smart messages (\S\ref{sec:nw:sm}). 
The reason we did not use eRPC as the networking layer of \odlib, is twofold. 
First, in eRPC, a broadcast requires a separate memcpy for each of the messages. 
In our setup that would result in multiple GBytes/s worth of unnecessary memcpying, for almost all protocols.
Secondly, eRPC would not allow us to use the multicast primitive.

\y{
Finally, G-DUR~\cite{Ardekani:2014b} is a generic middleware that enables the developers to implement and evaluate a large family of distributed transactional protocols.  %that leverage the Deferred Update Replication (DUR)approach.
G-DUR focuses on providing a substrate for transactional protocols that are based on the Deferred Update Replication (DUR) approach. In contrast, \odlib\ focuses on exploring the impact of modern hardware in strongly-consistent replication protocols.
}


\beginbsec{Analysis of replication protocols}
Ailijiang \etal~\cite{Ailijiang:2019} dissect the performance of strongly-consistent replication protocols. Their analysis is complimentary to ours, as they focused on latency and availability on wide-area-networks and geo-replication, while we focus on performance within the datacenter and over modern hardware.
% Renesse \etal~\cite{Renesse:2014}

\y{
\beginbsec{Modern Hardware}
\odlib\ investigates the interplay between protocol-level design decisions and three advances that are described as \emph{modern hardware}: many-core servers, user-level high-bandwidth networking  and high-capacity main memory.
Notably, Szekeres \etal~\cite{Szekeres:2020} also observe the importance of thread-scalability in the era of user-level networking, and propose the Zero-Coordination Principle a guideline to building thread-scalable replicated transactional storage systems. 
% This is complements \odlib, which focuses on the protocol-level actions that hinder thread-scalability.
Furthermore, recent work~\cite{Li:2016-NoPaxos, L1:2020, Zhu:2019, Li:2017, Jin:2017, Jin:2018, Firestone:2018} has investigated the impact of programmable hardware (FPGAs, smart NICs and switches) in deploying storage systems in the datacenter. Such programmable hardware can be used to accelerate the replication protocol. We believe that by uncovering the impact of protocol-level actions on performance our comparison of protocols can serve as a starting point for this endeavor, guiding both the selection of protocols to accelerate and the acceleration process itself.


}

\y{
\beginbsec{Skewed workloads}
Our evaluation does not investigate the sensitivity of replication protocols under a skewed workload (\eg zipfian distribution~\cite{Novakovic:2016}). This is not an oversight. 

It is possible to apply an optimization where reads and writes to the most popular keys (\ie the \qt{hot keys}) can be combined within each server by leveraging the fact that: 1) 
a server can efficiently keep track of the hot keys~\cite{S-Li:2016, Metwally:2005, Cormode:2008} and 2) at any given moment, a server is expected to be working on multiple requests for each of the hot keys.
This optimization turns skew from problem to opportunity. This is not a surprise: researches have repeatedly observed that skew is a form of locality, and as such it can be leveraged to increase performance~\cite{Priyank:2019,S-Li:2016, A&V:2018, L1:2020}.

Notably, the optimization is equally applicable to all \pnum\ protocols.
Consequently, 
evaluating the protocols without the optimization would paint a false picture, suggesting that protocols suffer under skew, when in reality they can thrive under it. 
However, 
the optimization will take a different shape for each protocol. Therefore, incorporating the optimization to all \pnum\ protocols will require substantial research  
and we leave it for future work.



}







