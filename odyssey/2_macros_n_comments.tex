%%%%%%%%%%%%%%%%%%%%%%
% Comments
%%%%%%%%%%%%%%%%%%%%%%
\newif\ifshowcomment
\showcommenttrue
% \showcommentfalse

\ifshowcomment

\newcommand{\todo}[1]{\noindent\textsf{\color{NavyBlue}{[{{\bf \scalebox{0.75}{\fbox{ToDo}}}: {\it#1}]}}}}
\newcommand{\newtext}[1]{\textcolor{blue}{#1}} % Added new texts
\newcommand{\modtext}[1]{\textcolor{red}{#1}}  % Modified texts
\newcommand{\boris}[1]{\noindent\textsf{\color{Violet}{[{{\bf \scalebox{0.75}{\fbox{Boris}}}: {\it#1}]}}}}
\newcommand{\vijay}[1]{\noindent\textsf{\color{purple}{[{{\bf \scalebox{0.75}{\fbox{Vijay}}}: {\it#1}]}}}}
\newcommand{\vasilis}[1]{\noindent\textsf{\color{orange}{[{{\bf \scalebox{0.75}{\fbox{Vasilis}}}: {\it#1}]}}}}
\newcommand{\arpit}[1]{\noindent\textsf{\color{magenta}{[{{\bf \scalebox{0.75}{\fbox{Arpit}}}: {\it#1}]}}}}
\newcommand{\antonis}[1]{\noindent\textsf{\color{OliveGreen}{[{{\bf \scalebox{0.75}{\fbox{Antonis}}}: {\it#1}]}}}}

\else
\newcommand{\newtext}[1]{#1} 
\newcommand{\modtext}[1]{#1}
\newcommand{\todo}[1]{}
\newcommand{\antonis}[1]{}
\newcommand{\boris}[1]{}
\newcommand{\vijay}[1]{}
\newcommand{\arpit}[1]{}
\newcommand{\vasilis}[1]{}
\fi

\newcommand{\y}[1]{#1}
% \newcommand{\y}[1]{{\color{blue} #1}\normalcolor}
% \newcommand{\y}[1]{{#1}}

%%%%%%%%%%%%%%%%%%%%%%%%%%%%%%%%%%%%%%
%%% CUSTOM COMMANDS
%%%%%%%%%%%%%%%%%%%%%%%%%%%%%%%%%%%%%%


%%%%%%%%%%%%%%%%%%%%%%
% Emphasized space-efficient bullets start
%%%%%%%%%%%%%%%%%%%%%%
\newcommand{\beginitem}[1]{\noindent $\succ$ \textit{#1}}



%%%%%%%%%%%%%%%%%%%%%%
% Floor and ceiling commands
%%%%%%%%%%%%%%%%%%%%%%
\newcommand{\floor}[1]{\lfloor #1 \rfloor}
\newcommand{\ceil}[1]{\lceil #1 \rceil}

%%%%%%%%%%%%%%%%%%%%%%
% Make full capitalized words less intrusive
%%%%%%%%%%%%%%%%%%%%%%
\newcommand{\CAP}[1]{\scalebox{0.85}{#1}}

%%%%%%%%%%%%%%%%%%%%%%
% circled character
%%%%%%%%%%%%%%%%%%%%%%
\newcommand*\circled[1]{\tikz[baseline=(char.base)]{
            \node[shape=circle,draw,inner sep=0.5pt] (char) {#1};}}


%%%%%%%%%%%%%%%%%%%%%%%%%
%% Squish lists
%% Usage: 
%%   \squishlist
%%     \item ..
%%   \squishend
%%%%%%%%%%%%%%%%%%%%%%%%%
\newcommand{\squishlist}{
 \begin{list}{$\bullet$}
  { \setlength{\itemsep}{2pt}
     \setlength{\parsep}{0pt}
     \setlength{\topsep}{2pt}
     \setlength{\partopsep}{0pt}
     \setlength{\leftmargin}{1em}
     \setlength{\labelwidth}{1em}
     \setlength{\labelsep}{0.5em} } 
}

\newcommand{\squishlistContrib}{ %
 \begin{list}{$\bullet$}
  { \setlength{\itemsep}{2pt}
     \setlength{\parsep}{0pt}
     \setlength{\topsep}{2pt}
     \setlength{\partopsep}{0pt}
     \setlength{\leftmargin}{1em}
     \setlength{\labelwidth}{1em}
     \setlength{\labelsep}{0.5em} }
}
\newcommand{\squishend}{ \end{list}  }


\newcommand{\squishenum}{\begin{enumerate}[itemsep=0.5pt,parsep=0pt,topsep=0pt,partopsep=0pt,leftmargin=1.5em,labelwidth=1em,labelsep=0.5em]{}}
\newcommand{\squishenumend}{\end{enumerate}}


%%% URLs
\newcommand\myurl[2]{\url{#1}}

\renewcommand{\floatpagefraction}{0.95}

\newcommand{\captionfonts}{\small}
\makeatletter  % Allow the use of @ in command names
\long\def\@makecaption#1#2{%
  \vskip 0.1in
  \sbox\@tempboxa{{\captionfonts #1: #2}}%
  \ifdim \wd\@tempboxa >\hsize
    {\captionfonts #1: #2\par}
  \else
    \hbox to\hsize{\hfil\box\@tempboxa\hfil}%
  \fi
  \vskip 0in}
\makeatother   % Cancel the effect of \makeatletter

% Theorems
\newtheorem{mydefinition}{Definition}
\newtheorem{mytheorem}{Theorem}
\newtheorem{definition}{Definition}
\newtheorem{theorem}{Theorem}

%%% Alignment
%\begin{center/flushright/flushleft}
%...
%\end{center/flushright/flushleft}


%% Margins
%  \usepackage[margin=0.5in]{geometry}

%%% Paragraphs and other breaks
% Paragraphs are separated by a blank line.
% You can force a new line using \\
% To force a new page, use \newpage or \clearpage

%%%% Other spacing
% Force a space using ∼
% Add space using \hspace{1in} or \vspace{1in}
% Fill space using \hfill or \vfill

\def\colorhl{\cellcolor[HTML]{C0C0C0}}
\def\hlrow{\rowcolor[HTML]{C0C0C0}}
\def\colorgrey{\cellcolor[HTML]{e6e6e6}}
\def\greyrow{\rowcolor[HTML]{e6e6e6}}
\def\custvspace{\vspace{0.4em}}
\newcommand{\qt}[1]{``#1''}
\def\colorhl{\cellcolor[HTML]{C0C0C0}} % highlight a the top cells of a table

%%%%%%%%%%%%%%%%%%%%%%
% Instead of using sub-subsections 
% use the following as an emphasized 
% first sentence of a new paragraph
%%%%%%%%%%%%%%%%%%%%%%
\newcommand{\beginbsec}[1]{\custvspace\noindent\textbf{#1.}}

\def\custevalvspace{\vspace{0.3em}}
\newcommand{\beginbseceval}[1]{\custevalvspace\noindent\textbf{$\succ$ #1.}}

\def\RDMA{RDMA}
\def\RMW{RMW}
\def\RMWs{RMWs}
\def\odlib{\emph{Odys\-sey}}
\def\pnum{ten}

\def\LTO{LTO}
\def\LPKO{LPKO}
\def\DTO{DTO}
\def\DPKO{DPKO}
\newcommand{\figref}[1]{Figure~\ref{#1}}
\newcommand{\secref}[1]{Section~\ref{#1}}
\newcommand{\tabref}[1]{Table~\ref{#1}}
%%%%%%%%%%%%%%%%%%%%
%%%% Latin Abbreviations 
%%%%%%%%%%%%%%%%%%%%
\def\etal{et~al.} % ``and others'', ``and co-workers''
\def\eg{e.g.,~} % ``for example''
\def\ie{i.e.,~} % ``that is'', ``in other words''
\def\etc{etc} % ``and other things'', ``and so forth''
\def\cf{cf.~} % ``compare''
\def\viz{viz.~} % ``namely", ``precisely''
\def\vs{vs.~} % ``against"

% \newcommand\eg[]{e.g., }
% \newcommand\ie[]{i.e., }
% \newcommand\eg[0]{e.g.\ }
% \newcommand\ie[0]{i.e.\ }
% \newcommand\et[0]{et al.\ }

%%%%%%%%%%%%%%%%%%%%
%%%% Spell check
%%%%%%%%%%%%%%%%%%%%

% if you want to spell-check your document, you can use the command-line aspell, hunspell (preferably), or ispell programs.
% E.g.:
%       ispell yourfile.tex
%   aspell --mode=tex -c yourfile.tex
%   hunspell -l -t -i utf-8 yourfile.tex

%% Word cound
%If you want to count words you can, again, use LyX or convert your LaTeX source to plain text and use, for example, UNIX wc command:
% detex yourfile | wc

%%%%%%%%%%%%%%%%%%%%
%% Hyphenated words
%%%%%%%%%%%%%%%%%%%%
%\hyphenation { hy-phen-a-tion mar-vel-ous-ly }