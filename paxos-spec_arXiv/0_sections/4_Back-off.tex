\section{The back-off mechanism} \label{sec:backoff}
We saw that if an RMW does not find the \kv~in \invalid~state, it does not grab it. Rather its \locentry~transitions to the \need~state. The \locentry~also stores the RMW that is currently holding the \kv~and its state. 

When the worker inspects the \locentry, if it finds it in \need~state, it attempts to grab the \kv. The \kv~can be grabbed if it's in \invalid~state. 
When the \kv~cannot be grabbed, the \locentry~notes the details of the \kv, its state, its current \propts~etc. Therefore, when the RMW fails repeatedly to grab the \kv, it takes note of whether the RMW that holds the \kv~has made any progress.
If nothing has changed since the the last time that the RMW attempted to grab the \kv, then a counter that is stored in the \locentry~is incremented. The counter measures the time in which the \kv~has not changed. When the counter reaches a threshold (pre-determined \eg at compile-time), then the RMW will attempt to steal or help the entry.
Otherwise, if there is progress since last inspected, then the counter goes to zero.

Essentially, this is a back-off mechanism. Recall that we can have thousands of RMW running concurrently. If a few of them are on the same \kv, then there is no chance of progress.
Instead, we limit the conflicting RMWs to one per machine, in the worst case. This is because a \kv~can be grabbed by a remote machine.

\custvspace

If the counter reaches the threshold, that typically means that the \kv~is held by a remote machine, which has failed. In this instance, if the \kv~is in \proped~state, an RMW can simply \qt{steal} the \kv, by using a higher \propts. The RMW will trigger propose broadcasts, and will transition its \locentry~to \proped~state, whereby it will wait for propose replies.

However, if the \kv~is in Accepted state, we can \emph{not} steal it. We must help it.