\section{Conclusion and Lessons Learned}
\label{sec:conclusion}

The goal of the paper is to uncover the impact of modern hardware on the performance of strongly-consistent replication protocols.
To this end, we presented \odlib, a framework that enables the fast development and deployment of replication protocols over modern hardware. 
Over \odlib, we built and evaluated \pnum\ protocols. 
Extrapolating their results to the entire design space through an informal taxonomy, we provided a characterization of strongly-consistent replication protocols.

On the system side, we experienced first-hand the necessity for a reliable, high-performance framework to design, build and deploy replication protocols. Without it, system-level bugs (networking, KVS etc.) become a black hole for developer time.
In hindsight, this is no surprise: clean interfaces that abstract orthogonal components have been the cornerstone of computer science. %
Nevertheless, 
we were pleasantly surprised to see that we can build and deploy a new protocol in two days (\S\ref{sec:why}). 

When it comes to protocol design, the overarching lesson is that the true limits of a protocol will be uncovered only when all artificially imposed bottlenecks have been removed. 
Plainly, this calls for highly-optimized, multi-threaded and \RDMA-enabled implementations. 
It is very telling that ZAB outperforms Classic Paxos (CP) by more than 2x when both are single-threaded, but the result is inverted when they are multi-threaded. 
The pseudo bottleneck of single-thread implementations conceal ZAB's inefficiencies while holding back CP's capabilities. Multi-threading removes the bottleneck, laying bare the true nature of the protocols.




