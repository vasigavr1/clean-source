\section{A Taxonomy of Replication Protocols}~\label{sec:tax}

This section serves two purposes.
First, we present a taxonomy of strongly-consistent replication protocols.
The taxonomy will not only inform our choice of protocols to implement and evaluate, but will also enable us to generalize the results of each protocol to its respective class.
Second, we describe the operation of various protocols, providing the background material necessary for the rest of this paper.
Before diving into the taxonomy we first offer three remarks on the protocols and the corresponding jargon.

\beginbsec{Remarks}
Firstly, note that a lot of the protocols that we discuss can also execute transactions. However, this work will view them solely through the lens of the read/write API, explaining how each protocol performs a read and a write to keys stored in the replicated KVS.

Secondly, note that the problem of performing a conditional write in an environment where machines can fail and network/processing delays are unbounded is equivalent to asynchronous consensus~\cite{Herlihy:2008}. 
This is why some of the protocols we are studying are known under the umbrella of \qt{consensus protocols}. 
However, in this work we cast a wider net, investigating the sensitivity of performance to relaxing the fault model or to 
downgrading the API from conditional writes to plain writes.
For that reason we refer to the protocols discussed in this paper with the general term \qt{strongly-consistent replication protocols}.

Finally, note that throughout this paper, when we refer to a \qt{local read}, we refer to an operation that is performed by a machine that knows it is in the configuration and hence reads from its local KVS.  

\subsection{Taxonomy}

Our taxonomy is split into four quadrants as shown in \tabref{tab:tax} based on two operational patterns: 1) leader-based (L) vs. decentralized (D) and 2) total order (TO) vs. per-key order (PKO). Consequently, there are four resulting classes of protocols:
\squishenum
\item \emph{\LTO}: leader-based total order 
\item \emph{\LPKO}: leader-based per-key order
\item \emph{\DTO}: decentralized total order 
\item \emph{\DPKO}: decentralized per-key order
\squishenumend

\custvspace
Total order implies that protocols create a total order of all writes across all keys and apply them to the KVS in that order. In contrast, per-key order mandates that protocols only enforce a total order of writes at a per-key basis. 
Note that this does not affect the consistency guarantees;
in both cases, protocols can offer lin.
Leader-based protocols utilize a single node (\ie a leader) to enforce the ordering of the writes, while decentralized protocols achieve the same effect in a distributed manner.

\begin{table}[t]
\centering
\resizebox{0.48\textwidth}{!}{%
\begin{tabular}{c|c|c|}
\hhline{~--} %
& 
\colorhl
\begin{tabular}[c]{@{}c@{}}

\textbf{Total order}

\end{tabular}      
& 
\colorhl
\begin{tabular}[c]{@{}c@{}}
\textbf{Per key order} \\ 

\end{tabular}
\\ \hline

\multicolumn{1}{|c|}{
\colorhl \textbf{
\begin{tabular}[c]{@{}c@{}}
Leader-\\ based 
\end{tabular} }}  & 
\begin{tabular}[c]{@{}c@{}}
    
    \textbf{Multi-Paxos}~\cite{Lamport:2001},
    \textbf{ZAB}~\cite{Hunt:2010, Reed:2008},\\
    VR~\cite{Oki:1988}, APUS~\cite{Wang:2017},
    DARE~\cite{Poke:2015},\\
    Raft~\cite{Ongaro:2014},
    Fast Paxos~\cite{Lamport:2006}
\end{tabular}
& 

\begin{tabular}[c]{@{}c@{}}  
    \textbf{CHT}~\cite{Chandra:2016},
    FGSMR~\cite{Liu:2020}, \\
    WPaxos~\cite{Ailijiang:2020}, 
    Primary-backup~\cite{Alsberg:1976},\\
    CR~\cite{VanRenesse:2004},
    \textbf{CRAQ}~\cite{Terrace:2009},
\end{tabular}                                                                                                                                                                                                \\ \hline
\multicolumn{1}{|c|}{
\colorhl \textbf{\begin{tabular}[c]{@{}c@{}}
Decentralized \\ (Leaderless)
\end{tabular}}}
& 

\begin{tabular}[c]{@{}c@{}}
Mencius~\cite{Mao:2008},
    \textbf{Derecho}~\cite{Jha:2019},\\
    AllConcur~\cite{Poke:2017}
\end{tabular}
& 
\begin{tabular}[c]{@{}c@{}}
\textbf{CP}~\cite{Lamport:1998},
RMW-Paxos\cite{Skrzypczak:2020},\\
CASPaxos\cite{Rystsov:2018}
 Gryff~\cite{Burke:2020},\\
 Generalized Paxos~\cite{Lamport:2005}, 
 EPaxos~\cite{Moraru:2013},\\
 
 Atlas~\cite{Enes:2020},
 \textbf{All-aboard Paxos}~\cite{Howard:2019} \\
 \textbf{ABD~\cite{Lynch:1997}}, 
 \textbf{Hermes}~\cite{A:2020} \\
\end{tabular} 

\\ \hline
\end{tabular}%




}
\caption{Taxonomy (implemented protocols are in bold)}
\label{tab:tax}
\end{table}






 





Why choose these two axes to categorize protocols?
We hypothesize that from a performance perspective, protocols must optimize for three metrics: 
1) thread-scalability: the protocol's ability to scale with more threads, 
2) load-balance: whether the work required to complete a request is evenly distributed among all nodes and 
3) the work-per-request ratio: the total cpu, network and memory resources required to complete a single request.

The classification is derived from the above three metrics.
Specifically, total order protocols---with or without a leader---struggle to achieve thread-scalability because applying writes in order requires coordination between the threads. 
Leader-based protocols struggle to achieve load balance as the leader tends to carry out most of the work required to execute a write. 
Both techniques (leader and total order)
help reduce the work-per-request ratio as they provide an easy way to serialize writes.
Conversely, protocols that are both per-key and leaderless tend to require a higher work-per-request ratio because the protocols must do additional work to serialize writes in a distributed manner.
We will substantiate these claims in our evaluation section (\S\ref{sec:ev}).





\subsection{Leader-based \& Total Order (\LTO)}\label{sec:tax:lto}
Protocols such as ZAB~\cite{Hunt:2010}, Multi-Paxos~\cite{Lamport:2001} and Raft~\cite{Ongaro:2014} serialize \emph{all} writes at the leader node, creating the total order. The leader executes the writes by proposing them to the rest of the nodes (dubbed \emph{followers}), typically in two broadcast rounds: a \emph{propose} round to which followers respond with an acknowledgement (ack), and a \emph{commit} round. All nodes must apply committed writes in their total order.  

\beginbsec{Reads}
A write is guaranteed to propagate to only a majority of nodes. The leader is the
only node that is guaranteed to be in that majority, and thus the only node guaranteed to know of the latest committed write for any key.
As such, the leader can always read locally.
Followers must send their reads to the leader, querying it for the latest value.

There are two possible relaxations that allow local reads in follower nodes, too. 
The first relaxation is to simply forego linearizability, conceding that reads may not return the latest write. This is tolerable 
for \LTO~protocols, because if writes are totally ordered, this relaxation downgrades consistency guarantees only mildly to Sequential Consistency~\cite{LevAri:2017}. ZAB subscribes to this practice.

The second relaxation that allows followers to read locally is to 
ensure that every write reaches all followers. 
Note that there is a downside in requiring that all writes propagate to \emph{all} nodes: even if one node fails, all writes block. We elaborate in \secref{sec:fail}.

\beginbsec{Choices}
To represent \LTO, we implement ZAB and Multi-Paxos (MP), capturing the difference between local reads (with relaxed consistency) and linearizable reads that must be sent to the leader node.


\subsection{Leader-based \& Per-key Order (\LPKO)}\label{sec:tax:lpko}

Protocols in this class use the leader node to only serialize writes \emph{to the same key}. Specifically, all writes are steered to the leader node, which simply ensures that writes to the same key are applied in the same order by all replicas. A typical example of this class is the CHT~\cite{Chandra:2016} protocol, where the leader executes writes in two rounds as described in the total order class. 
There are two possible optimizations protocols can employ.

The first is exemplified by Chain Replication (CR)~\cite{VanRenesse:2004}. In CR, the leader does not broadcast the writes to the followers; rather the nodes are organized in a chain, through which writes propagate from the head of the chain to its tail. The head node acts as the leader in that all writes have to be steered to it so that it serializes them. In our evaluation, we will see how this approach significantly---but not entirely---alleviates the load balance problem.

The second optimization also tackles load balance, by denoting that all nodes are leaders for a subset of the keys. For example, for a 5-node deployment the key space is partitioned five ways, where each node is denoted leader for only one of the partitions.
Notably, this is possible in \LPKO---but not \LTO---because the leader need not enforce an order across all writes. %

\beginbsec{Reads}
\LPKO~protocols can execute lin reads in the same manner as \LTO~protocols. %
When writes propagate to a majority of nodes, reads have to be propagated to the leader. When writes are guaranteed to propagate to all followers, reads can execute locally in all nodes. CHT and CRAQ~\cite{Terrace:2009}, an optimized variant of CR, both subscribe to this approach.

Finally, note that the option to propagate writes to a majority of nodes but execute reads locally by  downgrading consistency to SC (discussed for \LTO) is not available for per-key order protocols. Reading locally in this case would result in very weak guarantees (\ie Eventual Consistency~\cite{Vogels:2009}).

\beginbsec{Choices}
To represent \LPKO, we implement three protocols: CHT, CRAQ and a variant of CHT with multiple leaders, dubbed \emph{CHT-multi-ldr}. 
CHT represents the typical \LPKO~protocol, 
CRAQ captures the CR optimization for load balancing writes and %
finally, CHT-multi-ldr captures the optimization of denoting all nodes as leaders of a partition of the key space.
All three protocols read locally.





\subsection{Decentralized Total Order (\DTO)}\label{sec:tax:dto}
In \DTO\ protocols, the total order of writes is not created in a central location. Rather, 
there is typically a predetermined static allocation of write-ids to nodes. For example, 
all nodes know that the writes $0$ to $N - 1$ will be proposed and coordinated by node-0, the next $N$ writes (\ie $N$ to $2N - 1$) will be proposed by node-1 and so on.
Therefore, each node can calculate the place of each write in the total order based on its own node-id, without synchronizing with any other node. Then, the node broadcasts its writes along with their place in the total order. Typically a commit message is broadcast after gathering acks from a majority of the nodes.
Crucially, all nodes must apply the writes in the prescribed total order.
Derecho~\cite{Jha:2019}, AllConcur~\cite{Poke:2017} and Mencius~\cite{Mao:2008}, all belong to the \DTO~class.

\beginbsec{Reads}
Reads can be executed 
by allocating slots in the total order, similarly to writes.
Local reads are also possible, either by downgrading consistency guarantees to SC (similarly to \LTO), or by enforcing that all writes will propagate to all nodes.

\beginbsec{Choices}
To represent \DTO, we implement and evaluate Derecho. In order to get the upper bound of the \DTO~class, we implement the Derecho variant that executes reads locally, downgrading consistency guarantees to SC.

\subsection{Decentralized Per-key Order (\DPKO)}\label{sec:tax:dpko}
In the fourth and final quadrant, \DPKO~protocols %
agree on a per-key order of writes in a distributed manner. There is no central leader---rather any node can propose and coordinate a write. 
The most prominent example is Classic Paxos (CP)~\cite{Lamport:1998}.
Traditionally, CP has been regarded simply as a way to perform leader election so that Multi-Paxos can start executing.
However, recent proposals~\cite{Skrzypczak:2020, Rystsov:2018, V:2020}
have used CP to reach consensus on which node should be the next to perform a write at a per key basis.



Notably, CP extracts a steep price: it requires three broadcast rounds to complete (propose, accept and commit~\cite{Howard:2019}), each of which contains considerably more metadata than any other protocol we have discussed, while responding to a propose or accept is also very complicated, as there are various possible responses, depending on the state of other conflicting ongoing writes. Finally, depending on conflicts, CP may have to retry an unbounded number of times~\cite{Fischer:1985}. 

The source of CP's overhead stems from the combination of three constrains:
1) conflicting writes may be concurrently executing at all times \emph{and}
2) it is impossible to guarantee that a message will always be delivered to all nodes \emph{and}
3) writes are conditional (\ie RMWs).
Relaxing any of the constraints will significantly simplify the problem.
Consequently, there are three approaches to optimize CP, one for each constraint.
The first approach is exemplified by protocols such as EPaxos~\cite{Moraru:2013}, Atlas~\cite{Enes:2020} and All-aboard Paxos~\cite{Howard:2019}, which provide a fast path, where consensus can be achieved after two broadcast rounds (accept and commit), 
in the absence of conflicts, 
using CP as the fallback option when conflicts do occur.

The second approach is presented by Hermes~\cite{A:2020}, which, similarly to CR and CHT, 
enforces that a message will always be delivered to all nodes. With this guarantee, performing a write can be done in two lightweight broadcast rounds which are roughly equivalent to accept and commit. %


Finally, the third approach downgrades the API, offering plain writes instead of conditional writes.
Multi-writer ABD~\cite{Lynch:1997} is a variant of the ABD protocol~\cite{Attiya:1994} that exemplifies this approach. From now on, we refer to multi-writer ABD simply as ABD. 
A write in ABD requires two broadcast rounds that must reach a majority of nodes.

\beginbsec{Reads}
In \DPKO~protocols that do not guarantee that a write reaches all nodes, 
there is no master copy to read from. 
Therefore, to get the most recently committed write, a read must consult a majority of nodes~\cite{Charapko:2019}. The reads should then perform a second round to ensure that the write is committed to a majority of nodes, so that subsequent reads can also observe it. 
We refer to this as the \emph{ABD-read} as it was first proposed in the original ABD protocol~\cite{Attiya:1994}. Notably, if writes are guaranteed to reach all nodes, reads can be performed locally.

\beginbsec{Choices}
To represent \DPKO~we implement and evaluate four protocols: CP, All-aboard, Hermes and ABD. CP will provide a baseline. 
All-aboard shows the limit of CP while maintaining its availability guarantees.
Hermes will show us the performance gains possible when writes reach all nodes. ABD will showcase the performance difference between conditional and regular writes. 

Notably, instead of All-aboard, we could have selected EPaxos~\cite{Moraru:2013} (or its most recent variant, Atlas~\cite{Enes:2020}). EPaxos requires that nodes respond to accept messages with recent conflicting commands. This requires memory, compute and network resources to store, retrieve, reply and transmit an unbounded number of conflicting writes.
In contrast, All-aboard is a zero-cost optimization. 
Specifically, All-aboard leverages the Flexible Paxos~\cite{Howard:2018} theorem to shave off the first round (propose) and significantly reduce the size of the commit round, without incurring a counterweight cost.
The complete specification of our All-aboard implementation over CP can be found in~\cite{Paxos-spec}.





\subsection{The Impact on Availability}\label{sec:fail}
In this section, we discuss the implications of protocol design choices on the availability guarantees.


% \begin{table}[t]
% \centering
% \begin{tabular}{c|c|}
% \hhline{~-} %hline

% % \begin{tabular}[c]{@{}c@{}} Briefly unavailable\\ (Assuming N = 2f+1 nodes)\end{tabular} 
% & \multicolumn{1}{|c|}{\colorhl Protocols}
% \\ \hline

% \multicolumn{1}{|c|}{\colorhl Always available} &
% \multicolumn{1}{|c|}{\begin{tabular}[c]{@{}c@{}}CP, ABD,\\ All-aboard\end{tabular}}  
% \\ \hline

% \multicolumn{1}{|c|}{\colorhl 
% \begin{tabular}[c]{@{}c@{}}
% Briefly unavailable \\on a leader failure 
% \end{tabular}} &
% \multicolumn{1}{|c|}{ZAB, MP} 
% \\ \hline

% \multicolumn{1}{|c|}{\colorhl 
% \begin{tabular}[c]{@{}c@{}}
% Briefly unavailable \\on any node failure 
% \end{tabular}} &
% \multicolumn{1}{|c|}{
% \begin{tabular}[c]{@{}c@{}}
% Hermes, CRAQ, CHT, \\ CHT-multi-ldr, Derecho
% \end{tabular}} 
% \\ \hline
% \end{tabular}
% \caption{\y{The conditions under which protocols become unavailable for a duration of a predefined time-out}}
% \label{tab:avail}
% \end{table}


\begin{table}[t]
\centering
\resizebox{0.48\textwidth}{!}{%
\begin{tabular}{c|c|}
\hhline{~-} %hline

% \begin{tabular}[c]{@{}c@{}} Briefly unavailable\\ (Assuming N = 2f+1 nodes)\end{tabular} 
& \multicolumn{1}{|c|}{\colorhl Availability guarantees}
\\ \hline

 
\multicolumn{1}{|c|}{ \colorhl
\begin{tabular}[c]{@{}c@{}}
CP, ABD, All-aboard
\end{tabular}} & 

\multicolumn{1}{|c|}{ Always available}
\\ \hline


\multicolumn{1}{|c|}{ \colorhl ZAB, MP} &

\multicolumn{1}{|c|}{
\begin{tabular}[c]{@{}c@{}}
Unavailable for the duration of a \\ predefined time-out after the leader node fails 
\end{tabular}} 
\\ \hline


\multicolumn{1}{|c|}{ \colorhl
\begin{tabular}[c]{@{}c@{}}
Hermes, CRAQ, CHT, \\ CHT-multi-ldr, Derecho
\end{tabular}} &
\multicolumn{1}{|c|}{ 
\begin{tabular}[c]{@{}c@{}}
Unavailable for the duration of a \\ predefined time-out after any node fails
\end{tabular}} 

\\ \hline
\end{tabular}
}
\caption{\y{A summary of the availability guarantees of the \pnum\ protocols, with up to $f$ failures (with $2f + 1$ nodes). }}
\label{tab:avail}

\end{table}

CP, All-aboard and ABD offer the highest level of availability guarantees.
Specifically, they assume the possibility of: 
1) non-Byzantine machine and network failures; and 
2) unbounded delays in both processing and networking. 
Under these assumptions, as long as $N/2 + 1$ nodes remain alive, responsive and connected, these three protocols will operate without interruption, \ie they will remain available.
The rest of the protocols that we have selected 
make design choices that downgrade these availability guarantees.


Specifically, leader-based protocols (ZAB, MP, CRAQ, CHT and CHT-multi-ldr)  will block if the leader becomes unresponsive.
Similarly, assuming that writes always reach all nodes (as in Hermes, CRAQ, CHT, and CHT-multi-ldr) results in blocking if any node becomes unresponsive.
Note that assuming that writes reach all nodes is a prerequisite for  linearizable local reads. Therefore, lin local reads can only be implemented at the expense of availability.
Finally, Derecho assumes that every node makes use of their pre-allocated slots in the total order in a timely manner. If any node is slow to broadcast new writes, then all nodes will block.
\y{\tabref{tab:avail} provides a brief summary of the availability guarantees of the \pnum\ protocols.}


In all the above cases, a failure causes blocking for the duration of a predefined time-out. Expending this time-out will trigger a recovery action (\eg leader election, reconfiguration etc.). Once recovery is complete, operation can resume. 
The unavailability period is the sum of the length of the time-out plus the latency of the recovery action. 

This work provides a detailed performance analysis of replication protocols without delving into the nuances of availability. However, having pointed to the choices that come at the expense of availability, we enable the operator to select (or design) the protocol that best fits their needs.













