\section{Related Work}
\label{sec:related}

\beginbsec{Related Frameworks}
Similarly to \odlib, 
Paxi~\cite{Ailijiang:2019} offers a rich interface that enables the fast development of replication protocols. However, Paxi is neither multi-threaded nor \RDMA-enabled.
eRPC~\cite{Kalia:2019} is a general-purpose networking framework offering \RDMA-based RPCs, similarly to \odlib.
However, \odlib\ also provides functionality tailored for replication protocols, such as the smart messages (\S\ref{sec:nw:sm}). 
The reason we did not use eRPC as the networking layer of \odlib, is twofold. 
First, in eRPC, a broadcast requires a separate memcpy for each of the messages. 
In our setup that would result in multiple GBytes/s worth of unnecessary memcpying, for almost all protocols.
Secondly, eRPC would not allow us to use the multicast primitive.

\y{
Finally, G-DUR~\cite{Ardekani:2014b} is a generic middleware that enables the developers to implement and evaluate a large family of distributed transactional protocols.  %
G-DUR focuses on providing a substrate for transactional protocols that are based on the Deferred Update Replication (DUR) approach. In contrast, \odlib\ focuses on exploring the impact of modern hardware in strongly-consistent replication protocols.
}


\beginbsec{Analysis of replication protocols}
Ailijiang \etal~\cite{Ailijiang:2019} dissect the performance of strongly-consistent replication protocols. Their analysis is complimentary to ours, as they focused on latency and availability on wide-area-networks and geo-replication, while we focus on performance within the datacenter and over modern hardware.

\y{
\beginbsec{Modern Hardware}
\odlib\ investigates the interplay between protocol-level design decisions and three advances that are described as \emph{modern hardware}: many-core servers, user-level high-bandwidth networking  and high-capacity main memory.
Notably, Szekeres \etal~\cite{Szekeres:2020} also observe the importance of thread-scalability in the era of user-level networking, and propose the Zero-Coordination Principle a guideline to building thread-scalable replicated transactional storage systems. 
Furthermore, recent work~\cite{Li:2016-NoPaxos, L1:2020, Zhu:2019, Li:2017, Jin:2017, Jin:2018, Firestone:2018} has investigated the impact of programmable hardware (FPGAs, smart NICs and switches) in deploying storage systems in the datacenter. Such programmable hardware can be used to accelerate the replication protocol. We believe that by uncovering the impact of protocol-level actions on performance our comparison of protocols can serve as a starting point for this endeavor, guiding both the selection of protocols to accelerate and the acceleration process itself.


}

% \label{sec:discussion}
\y{
\beginbsec{Skewed workloads}
Our evaluation does not investigate the sensitivity of replication protocols under a skewed workload (\eg zipfian distribution~\cite{Novakovic:2016}). This is not an oversight. 

It is possible to apply an optimization where reads and writes to the most popular keys (\ie the \qt{hot keys}) can be combined within each server by leveraging the fact that: 1) 
a server can efficiently keep track of the hot keys~\cite{S-Li:2016, Metwally:2005, Cormode:2008} and 2) at any given moment, a server is expected to be working on multiple requests for each of the hot keys.
% seeing as at any given moment, a server is expected to be working on multiple requests for each of the most popular objects, which it can efficiently keep track  Keeping track of the hottest objects   is ~\cite{S-Li:2016, Metwally:2005, Cormode:2008}
This optimization turns skew from problem to opportunity. This is not a surprise: researches have repeatedly observed that skew is a form of locality, and as such it can be leveraged to increase performance~\cite{Priyank:2019,S-Li:2016, A&V:2018, L1:2020}.
% This optimization turns skew from problem to opportunity and is equally applicable to all \pnum\ protocols.

Notably, the optimization is equally applicable to all \pnum\ protocols.
Consequently, 
% On the one hand, 
evaluating the protocols without the optimization would paint a false picture, suggesting that protocols suffer under skew, when in reality they can thrive under it. 
However, 
% On the other hand,
%adding the optimization to all \pnum\ protocols merits its own dedicated paper, as
the optimization will take a different shape for each protocol. Therefore, incorporating the optimization to all \pnum\ protocols will require substantial research  
%Such research effort is not within the scope of this paper which focuses on exploring the interplay between protocol-level design decisions and modern hardware.
and we leave it for future work.
% For completeness, we briefly describe the optimization below.


% For this reason, we believe that the investigation of the sensitivity of replication protocol to skew, merits its own dedicated paper. 
% Therefore, we select to exclude skew from our study, focusing instead on the interplay between protocol-level design decisions and modern hardware. 

}
\begin{comment}


\beginbsec{Optimization}
Assume a configuration where in each server there are 20 workers running 50 sessions each. At any given time a server can be executing up to a thousand requests. From a consistency standpoint these requests are completely \emph{independent}, as they originate from different sessions.
Under a zipfian distribution, a substantial percentage of these requests will access only a handful of the keys, \ie the hottest ones. As a rule of thumb, in a KVS that stores a million keys, under a zipfian distribution with $\alpha = 0.99$, (a typical number~\cite{Dragojevic:2014, Hong:2013, Novakovic:2019}) roughly 16\% of accesses refer to the 20 hottest keys.
Furthermore, a number of efficient techniques can be used to identify and keep track of the hottest keys with very low overhead~\cite{S-Li:2016, Metwally:2005, Cormode:2008}.
Therefore, it is possible for a server to combine independent requests to the hottest objects. For example, assume that 16\% of accesses refer to the 20 hottest keys and a server runs a thousand unrelated requests. The server can combine 160 of the requests -- those to the 20 hottest keys -- in only 20, one for each key.
Consistency will not be violated because the requests are independent, as they come from different sessions.

% We prototyped this optimization in Hermes to gauge its potential. With the optimization and found out that it can increase the performance

% Applying this optimization turns skew from problem to opportunity. This is not a surprise: skew is a form of locality, an insight which researches have repeatedly leverage  to turn skew into an advantage~\cite{Priyank:2019,S-Li:2016, A&V:2018, L1:2020}.
\end{comment}




\begin{comment}
\subsection{Why not skew?}

We strongly believe that skew merits a separate, dedicated paper. Here is why.

A protocol-level optimization that Hermes employs (coalescing writes/reads to hot keys across servers) turns skew from problem to opportunity. This optimization should be equally applicable to all ten protocols.

On the one hand, evaluating the protocols without the optimization would paint a false picture. On the other hand adding the optimization to all ten protocols will require significant research/engineering as the optimization will take a different shape for each protocol.


Because protocols can be optimized to leverage skew.
Any server will be executing thousands of RMWs any given point.
With skew a substantial percentage of them will be for the most popular data item.
What's to stop the server from coalescing the RMWs, and execute 10s or hundreds of RMWs, by simply executing one?
Absent such optimizations, the conclusions drawn from skewed workloads will not be as useful.

\subsection{Why throughput?}
Throughput shows the load a system can handle.
It also says a lot about latency: it describes under what load a system will have low latency.
\end{comment}

